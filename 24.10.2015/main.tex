\documentclass[a4paper,20pt,landscape]{extreport}

\usepackage{../main}

\begin{document}

\definecolor{keyword}{rgb}{0.2,0.2,0.8}

\lstset{ %
language=C++,
basicstyle=\small\ttfamily, % размер и начертание шрифта для подсветки кода
numbers=left,               % где поставить нумерацию строк (слева\справа)
numberstyle=\small,           % размер шрифта для номеров строк
stepnumber=1,                   % размер шага между двумя номерами строк
numbersep=8pt,                % как далеко отстоят номера строк от подсвечиваемого кода
backgroundcolor=\color{white}, % цвет фона подсветки - используем \usepackage{color}
keywordstyle=\color{keyword}\bfseries,
stringstyle=\color{red},
showspaces=false,            % показывать или нет пробелы специальными отступами
showstringspaces=false,      % показывать или нет пробелы в строках
showtabs=false,             % показывать или нет табуляцию в строках
frame=none,              % рисовать рамку вокруг кода
tabsize=4,                 % размер табуляции по умолчанию равен 2 пробелам
captionpos=t,              % позиция заголовка вверху [t] или внизу [b] 
breaklines=true,           % автоматически переносить строки (да\нет)
breakatwhitespace=false, % переносить строки только если есть пробел
escapeinside={\%*}{*)}   % если нужно добавить комментарии в коде
}

\begin{titlepage}
\newpage
\ThisCenterWallPaper{1.0}{../fon.jpg}

~\vspace{5cm}

\Large{Разбор задач олимпиады}

\Large{<<Третья командная олимпиада>>}

\Large{24.10.2015г.}

\end{titlepage}


{\large Задача А. Треугольник}
\\
\\
Разбор от Олега
\newpage
{\large Решение на С++}
\\
\\
\lstinputlisting[language=C++]{A.cpp}

\newpage
{\large Задача B. Цепочка слов} 
\\
\\
Разбор от Олега
\newpage
{\large Решение на С++}
\\
\\
\lstinputlisting[language=C++]{B.cpp}

\newpage
{\large Задача C. Игра <<Flip-Flop>>} 
\\
\\
Разбор от Олега
\newpage
{\large Решение на С++}
\\
\\
\lstinputlisting[language=C++]{C.cpp}

\newpage
{\large Задача D. Голодный ферзь - 2} 
\\
\\
Так как координаты пешек могут очень большими (по сравнению с их количеством), то первым делом проведём операцию сжатия координат (с помощью сортировки, удаления дубликатов и бинарного поиска).

После этого можно завести четыре массива множеств, которые хранят данные о пешках по горизонтали, вертикали и двум диагоналям. Например для сета, отвечающего за конкретную горизонталь будем хранить координату вертикали пешки. С помощью этих данных можно будет быстро отвечать на вопрос <<есть ли пешки на отрезке?>>, вызвав $upper\_bound$ для левого конца. Если эта функция вышла за границу правого конца, значит между этими точками других пешек нет. По аналогии организованы и все остальные сеты.

Не забываем удалять пешку из всех сетов после её <<поедания>>.

\newpage
{\large Решение на С++}
\\
\\
\lstinputlisting[language=C++]{D.cpp}

\newpage
{\large Задача E. Забег} 
\\
\\
Разбор от Олега
\newpage
{\large Решение на С++}
\\
\\
\lstinputlisting[language=C++]{E.cpp}

\newpage
{\large Задача F. Формула} 
\\
\\
Разбор от Олега
\newpage
{\large Решение на С++}
\\
\\
\lstinputlisting[language=C++]{F.cpp}

\newpage
{\large Задача G. Беспризорник} 
\\
\\
Разбор от Олега
\newpage
{\large Решение на С++}
\\
\\
\lstinputlisting[language=C++]{G.cpp}

\newpage
{\large Задача H. Квадратный корень} 
\\
\\
Разбор от Олега
\newpage
{\large Решение на С++}
\\
\\
\lstinputlisting[language=C++]{H.cpp}

\newpage
{\large Задача I. Студенты} 
\\
\\
Разбор от Олега
\newpage
{\large Решение на С++}
\\
\\
\lstinputlisting[language=C++]{I.cpp}

\newpage
{\large Задача J. Слова} 
\\
\\
Разбор от Олега
\newpage
{\large Решение на С++}
\\
\\
\lstinputlisting[language=C++]{J.cpp}


\end{document}
