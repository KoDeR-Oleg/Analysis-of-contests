{\large Задача D. Голодный ферзь - 2} 
\\
\\
Так как координаты пешек могут очень большими (по сравнению с их количеством), то первым делом проведём операцию сжатия координат (с помощью сортировки, удаления дубликатов и бинарного поиска).

После этого можно завести четыре массива множеств, которые хранят данные о пешках по горизонтали, вертикали и двум диагоналям. Например для сета, отвечающего за конкретную горизонталь будем хранить координату вертикали пешки. С помощью этих данных можно будет быстро отвечать на вопрос <<есть ли пешки на отрезке?>>, вызвав $upper\_bound$ для левого конца. Если эта функция вышла за границу правого конца, значит между этими точками других пешек нет. По аналогии организованы и все остальные сеты.

Не забываем удалять пешку из всех сетов после её <<поедания>>.

\newpage
{\large Решение на С++}
\\
\\
\lstinputlisting[language=C++]{D.cpp}
