\documentclass[a4paper,20pt,landscape]{extreport}

\usepackage{../main}

\begin{document}

\definecolor{keyword}{rgb}{0.2,0.2,0.8}

\lstset{ %
language=C++,
basicstyle=\small\ttfamily, % размер и начертание шрифта для подсветки кода
numbers=left,               % где поставить нумерацию строк (слева\справа)
numberstyle=\small,           % размер шрифта для номеров строк
stepnumber=1,                   % размер шага между двумя номерами строк
numbersep=8pt,                % как далеко отстоят номера строк от подсвечиваемого кода
backgroundcolor=\color{white}, % цвет фона подсветки - используем \usepackage{color}
keywordstyle=\color{keyword}\bfseries,
stringstyle=\color{red},
showspaces=false,            % показывать или нет пробелы специальными отступами
showstringspaces=false,      % показывать или нет пробелы в строках
showtabs=false,             % показывать или нет табуляцию в строках
frame=none,              % рисовать рамку вокруг кода
tabsize=4,                 % размер табуляции по умолчанию равен 2 пробелам
captionpos=t,              % позиция заголовка вверху [t] или внизу [b] 
breaklines=true,           % автоматически переносить строки (да\нет)
breakatwhitespace=false, % переносить строки только если есть пробел
escapeinside={\%*}{*)}   % если нужно добавить комментарии в коде
}

\begin{titlepage}
\newpage
\ThisLRCornerWallPaper{1.0}{../fon.jpg}
~\vspace{5cm}

\Large{Разбор задач олимпиады}

\Large{<<Третья личная олимпиада>>}

\Large{17.10.2015г.}

\end{titlepage}


{\large Задача А. Наименьшее общее кратное}
\\
\\
При ограничениях $x$ и $y$ до $10^3$ их наименьшее общее кратное не будет превосходить $10^6$. А это значит, что можно проверить все числа от 1 до $10^6$ и найти первое такое, которое бы делилось и на $x$, и на $y$.

В коде ниже приведён более совершенный алгоритм для нахждения наибольшего общего делителя--- алгоритм Евклида. Более подробно почитать про алгоритм Евклида можно в сборнике <<Справочник спортивного программиста>>: https://vk.com/topic-57951380\_30084443.

\newpage
{\large Задача B. Постройка дорог} 
\\
\\
Разбор от Олега

\newpage
{\large Решение на С++}
\\
\\
\lstinputlisting[language=C++]{B.cpp}

\newpage
{\large Задача C. Длиннейшая общая подпара} 
\\
\\
Разбор от меня

\newpage
{\large Решение на С++}
\\
\\
\lstinputlisting[language=C++]{C.cpp}

\newpage
{\large Задача D. Пастух} 
\\
\\
Разбор от меня
\newpage
{\large Решение на С++}
\\
\\
\lstinputlisting[language=C++]{D.cpp}


\end{document}
