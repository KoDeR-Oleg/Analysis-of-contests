{\large Задача C. Длиннейшая общая подпара} 
\\
\\
Для решения задачи воспользуемся методом динамического программирования, а именно, классическим алгоритмом поиска наибольшей общей подстроки. Применим его два раза: для всех префиксов и для всех суффиксов заданных строк. После чего необходимо выбрать такую пару индексов $(i, j)$, чтобы суммарная длина наибольших общих подстрок у префиксов $\xi[0..i]$ и $\eta[0..j]$ и у суффиксов $\xi[i+1..n]$ и $\eta[j+1..m]$ была максимальной. Это можно сделать простым перебором всех пар. Время работы алгоритма составит $O(N^2)$.
\\
\\
Заметим, что данную задачу можно решить за время $O(N \log N)$ с помощью суффиксного автомата и дерева Фенвика.

\newpage
{\large Решение на С++}
\\
\\
\lstinputlisting[language=C++]{C.cpp}
