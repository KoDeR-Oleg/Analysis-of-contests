{\large Задача B. Постройка дорог} 
\\
\\
Для решения задачи выведем формулу для произвольного $n$.
Заметим, что имея ответ для $n$ ($ans[n]$), ответ для $n+1$ будет равен $3^{n} \cdot ans[n]$.
Действительно, $n+1$ город связан одним их трех способов (в одну сторону, в другую сторону, не связан) с каждым из $n$ городов, а $n$ городов связаны друг с другом $ans[n]$ количеством способов. 

Найдем итоговую степень тройки. Она будет вида $(n-1) + (n-2) ... 1$.

Это сумма арифметической прогрессии $s = (n-1) \cdot ((n-1) + 1)/2 = (n-1) \cdot n/2$.

Очевидно, что ответ будет большой для заданных ограничений, поэтому будем считать используя длинную арифметику.

В массиве $a$ будем хранить $i$-тую цифру ответа, считая с конца. Изначально $a[1] = 1$.

Умножим $s$ раз число на 3. Итоговое число и будет ответом.


\newpage
{\large Решение на С++}
\\
\\
\lstinputlisting[language=C++]{B.cpp}
